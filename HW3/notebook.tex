
% Default to the notebook output style

    


% Inherit from the specified cell style.




    
\documentclass[11pt]{article}

    
    
    \usepackage[T1]{fontenc}
    % Nicer default font (+ math font) than Computer Modern for most use cases
    \usepackage{mathpazo}

    % Basic figure setup, for now with no caption control since it's done
    % automatically by Pandoc (which extracts ![](path) syntax from Markdown).
    \usepackage{graphicx}
    % We will generate all images so they have a width \maxwidth. This means
    % that they will get their normal width if they fit onto the page, but
    % are scaled down if they would overflow the margins.
    \makeatletter
    \def\maxwidth{\ifdim\Gin@nat@width>\linewidth\linewidth
    \else\Gin@nat@width\fi}
    \makeatother
    \let\Oldincludegraphics\includegraphics
    % Set max figure width to be 80% of text width, for now hardcoded.
    \renewcommand{\includegraphics}[1]{\Oldincludegraphics[width=.8\maxwidth]{#1}}
    % Ensure that by default, figures have no caption (until we provide a
    % proper Figure object with a Caption API and a way to capture that
    % in the conversion process - todo).
    \usepackage{caption}
    \DeclareCaptionLabelFormat{nolabel}{}
    \captionsetup{labelformat=nolabel}

    \usepackage{adjustbox} % Used to constrain images to a maximum size 
    \usepackage{xcolor} % Allow colors to be defined
    \usepackage{enumerate} % Needed for markdown enumerations to work
    \usepackage{geometry} % Used to adjust the document margins
    \usepackage{amsmath} % Equations
    \usepackage{amssymb} % Equations
    \usepackage{textcomp} % defines textquotesingle
    % Hack from http://tex.stackexchange.com/a/47451/13684:
    \AtBeginDocument{%
        \def\PYZsq{\textquotesingle}% Upright quotes in Pygmentized code
    }
    \usepackage{upquote} % Upright quotes for verbatim code
    \usepackage{eurosym} % defines \euro
    \usepackage[mathletters]{ucs} % Extended unicode (utf-8) support
    \usepackage[utf8x]{inputenc} % Allow utf-8 characters in the tex document
    \usepackage{fancyvrb} % verbatim replacement that allows latex
    \usepackage{grffile} % extends the file name processing of package graphics 
                         % to support a larger range 
    % The hyperref package gives us a pdf with properly built
    % internal navigation ('pdf bookmarks' for the table of contents,
    % internal cross-reference links, web links for URLs, etc.)
    \usepackage{hyperref}
    \usepackage{longtable} % longtable support required by pandoc >1.10
    \usepackage{booktabs}  % table support for pandoc > 1.12.2
    \usepackage[inline]{enumitem} % IRkernel/repr support (it uses the enumerate* environment)
    \usepackage[normalem]{ulem} % ulem is needed to support strikethroughs (\sout)
                                % normalem makes italics be italics, not underlines
    

    
    
    % Colors for the hyperref package
    \definecolor{urlcolor}{rgb}{0,.145,.698}
    \definecolor{linkcolor}{rgb}{.71,0.21,0.01}
    \definecolor{citecolor}{rgb}{.12,.54,.11}

    % ANSI colors
    \definecolor{ansi-black}{HTML}{3E424D}
    \definecolor{ansi-black-intense}{HTML}{282C36}
    \definecolor{ansi-red}{HTML}{E75C58}
    \definecolor{ansi-red-intense}{HTML}{B22B31}
    \definecolor{ansi-green}{HTML}{00A250}
    \definecolor{ansi-green-intense}{HTML}{007427}
    \definecolor{ansi-yellow}{HTML}{DDB62B}
    \definecolor{ansi-yellow-intense}{HTML}{B27D12}
    \definecolor{ansi-blue}{HTML}{208FFB}
    \definecolor{ansi-blue-intense}{HTML}{0065CA}
    \definecolor{ansi-magenta}{HTML}{D160C4}
    \definecolor{ansi-magenta-intense}{HTML}{A03196}
    \definecolor{ansi-cyan}{HTML}{60C6C8}
    \definecolor{ansi-cyan-intense}{HTML}{258F8F}
    \definecolor{ansi-white}{HTML}{C5C1B4}
    \definecolor{ansi-white-intense}{HTML}{A1A6B2}

    % commands and environments needed by pandoc snippets
    % extracted from the output of `pandoc -s`
    \providecommand{\tightlist}{%
      \setlength{\itemsep}{0pt}\setlength{\parskip}{0pt}}
    \DefineVerbatimEnvironment{Highlighting}{Verbatim}{commandchars=\\\{\}}
    % Add ',fontsize=\small' for more characters per line
    \newenvironment{Shaded}{}{}
    \newcommand{\KeywordTok}[1]{\textcolor[rgb]{0.00,0.44,0.13}{\textbf{{#1}}}}
    \newcommand{\DataTypeTok}[1]{\textcolor[rgb]{0.56,0.13,0.00}{{#1}}}
    \newcommand{\DecValTok}[1]{\textcolor[rgb]{0.25,0.63,0.44}{{#1}}}
    \newcommand{\BaseNTok}[1]{\textcolor[rgb]{0.25,0.63,0.44}{{#1}}}
    \newcommand{\FloatTok}[1]{\textcolor[rgb]{0.25,0.63,0.44}{{#1}}}
    \newcommand{\CharTok}[1]{\textcolor[rgb]{0.25,0.44,0.63}{{#1}}}
    \newcommand{\StringTok}[1]{\textcolor[rgb]{0.25,0.44,0.63}{{#1}}}
    \newcommand{\CommentTok}[1]{\textcolor[rgb]{0.38,0.63,0.69}{\textit{{#1}}}}
    \newcommand{\OtherTok}[1]{\textcolor[rgb]{0.00,0.44,0.13}{{#1}}}
    \newcommand{\AlertTok}[1]{\textcolor[rgb]{1.00,0.00,0.00}{\textbf{{#1}}}}
    \newcommand{\FunctionTok}[1]{\textcolor[rgb]{0.02,0.16,0.49}{{#1}}}
    \newcommand{\RegionMarkerTok}[1]{{#1}}
    \newcommand{\ErrorTok}[1]{\textcolor[rgb]{1.00,0.00,0.00}{\textbf{{#1}}}}
    \newcommand{\NormalTok}[1]{{#1}}
    
    % Additional commands for more recent versions of Pandoc
    \newcommand{\ConstantTok}[1]{\textcolor[rgb]{0.53,0.00,0.00}{{#1}}}
    \newcommand{\SpecialCharTok}[1]{\textcolor[rgb]{0.25,0.44,0.63}{{#1}}}
    \newcommand{\VerbatimStringTok}[1]{\textcolor[rgb]{0.25,0.44,0.63}{{#1}}}
    \newcommand{\SpecialStringTok}[1]{\textcolor[rgb]{0.73,0.40,0.53}{{#1}}}
    \newcommand{\ImportTok}[1]{{#1}}
    \newcommand{\DocumentationTok}[1]{\textcolor[rgb]{0.73,0.13,0.13}{\textit{{#1}}}}
    \newcommand{\AnnotationTok}[1]{\textcolor[rgb]{0.38,0.63,0.69}{\textbf{\textit{{#1}}}}}
    \newcommand{\CommentVarTok}[1]{\textcolor[rgb]{0.38,0.63,0.69}{\textbf{\textit{{#1}}}}}
    \newcommand{\VariableTok}[1]{\textcolor[rgb]{0.10,0.09,0.49}{{#1}}}
    \newcommand{\ControlFlowTok}[1]{\textcolor[rgb]{0.00,0.44,0.13}{\textbf{{#1}}}}
    \newcommand{\OperatorTok}[1]{\textcolor[rgb]{0.40,0.40,0.40}{{#1}}}
    \newcommand{\BuiltInTok}[1]{{#1}}
    \newcommand{\ExtensionTok}[1]{{#1}}
    \newcommand{\PreprocessorTok}[1]{\textcolor[rgb]{0.74,0.48,0.00}{{#1}}}
    \newcommand{\AttributeTok}[1]{\textcolor[rgb]{0.49,0.56,0.16}{{#1}}}
    \newcommand{\InformationTok}[1]{\textcolor[rgb]{0.38,0.63,0.69}{\textbf{\textit{{#1}}}}}
    \newcommand{\WarningTok}[1]{\textcolor[rgb]{0.38,0.63,0.69}{\textbf{\textit{{#1}}}}}
    
    
    % Define a nice break command that doesn't care if a line doesn't already
    % exist.
    \def\br{\hspace*{\fill} \\* }
    % Math Jax compatability definitions
    \def\gt{>}
    \def\lt{<}
    % Document parameters
    \title{HW3}
    
    
    

    % Pygments definitions
    
\makeatletter
\def\PY@reset{\let\PY@it=\relax \let\PY@bf=\relax%
    \let\PY@ul=\relax \let\PY@tc=\relax%
    \let\PY@bc=\relax \let\PY@ff=\relax}
\def\PY@tok#1{\csname PY@tok@#1\endcsname}
\def\PY@toks#1+{\ifx\relax#1\empty\else%
    \PY@tok{#1}\expandafter\PY@toks\fi}
\def\PY@do#1{\PY@bc{\PY@tc{\PY@ul{%
    \PY@it{\PY@bf{\PY@ff{#1}}}}}}}
\def\PY#1#2{\PY@reset\PY@toks#1+\relax+\PY@do{#2}}

\expandafter\def\csname PY@tok@w\endcsname{\def\PY@tc##1{\textcolor[rgb]{0.73,0.73,0.73}{##1}}}
\expandafter\def\csname PY@tok@c\endcsname{\let\PY@it=\textit\def\PY@tc##1{\textcolor[rgb]{0.25,0.50,0.50}{##1}}}
\expandafter\def\csname PY@tok@cp\endcsname{\def\PY@tc##1{\textcolor[rgb]{0.74,0.48,0.00}{##1}}}
\expandafter\def\csname PY@tok@k\endcsname{\let\PY@bf=\textbf\def\PY@tc##1{\textcolor[rgb]{0.00,0.50,0.00}{##1}}}
\expandafter\def\csname PY@tok@kp\endcsname{\def\PY@tc##1{\textcolor[rgb]{0.00,0.50,0.00}{##1}}}
\expandafter\def\csname PY@tok@kt\endcsname{\def\PY@tc##1{\textcolor[rgb]{0.69,0.00,0.25}{##1}}}
\expandafter\def\csname PY@tok@o\endcsname{\def\PY@tc##1{\textcolor[rgb]{0.40,0.40,0.40}{##1}}}
\expandafter\def\csname PY@tok@ow\endcsname{\let\PY@bf=\textbf\def\PY@tc##1{\textcolor[rgb]{0.67,0.13,1.00}{##1}}}
\expandafter\def\csname PY@tok@nb\endcsname{\def\PY@tc##1{\textcolor[rgb]{0.00,0.50,0.00}{##1}}}
\expandafter\def\csname PY@tok@nf\endcsname{\def\PY@tc##1{\textcolor[rgb]{0.00,0.00,1.00}{##1}}}
\expandafter\def\csname PY@tok@nc\endcsname{\let\PY@bf=\textbf\def\PY@tc##1{\textcolor[rgb]{0.00,0.00,1.00}{##1}}}
\expandafter\def\csname PY@tok@nn\endcsname{\let\PY@bf=\textbf\def\PY@tc##1{\textcolor[rgb]{0.00,0.00,1.00}{##1}}}
\expandafter\def\csname PY@tok@ne\endcsname{\let\PY@bf=\textbf\def\PY@tc##1{\textcolor[rgb]{0.82,0.25,0.23}{##1}}}
\expandafter\def\csname PY@tok@nv\endcsname{\def\PY@tc##1{\textcolor[rgb]{0.10,0.09,0.49}{##1}}}
\expandafter\def\csname PY@tok@no\endcsname{\def\PY@tc##1{\textcolor[rgb]{0.53,0.00,0.00}{##1}}}
\expandafter\def\csname PY@tok@nl\endcsname{\def\PY@tc##1{\textcolor[rgb]{0.63,0.63,0.00}{##1}}}
\expandafter\def\csname PY@tok@ni\endcsname{\let\PY@bf=\textbf\def\PY@tc##1{\textcolor[rgb]{0.60,0.60,0.60}{##1}}}
\expandafter\def\csname PY@tok@na\endcsname{\def\PY@tc##1{\textcolor[rgb]{0.49,0.56,0.16}{##1}}}
\expandafter\def\csname PY@tok@nt\endcsname{\let\PY@bf=\textbf\def\PY@tc##1{\textcolor[rgb]{0.00,0.50,0.00}{##1}}}
\expandafter\def\csname PY@tok@nd\endcsname{\def\PY@tc##1{\textcolor[rgb]{0.67,0.13,1.00}{##1}}}
\expandafter\def\csname PY@tok@s\endcsname{\def\PY@tc##1{\textcolor[rgb]{0.73,0.13,0.13}{##1}}}
\expandafter\def\csname PY@tok@sd\endcsname{\let\PY@it=\textit\def\PY@tc##1{\textcolor[rgb]{0.73,0.13,0.13}{##1}}}
\expandafter\def\csname PY@tok@si\endcsname{\let\PY@bf=\textbf\def\PY@tc##1{\textcolor[rgb]{0.73,0.40,0.53}{##1}}}
\expandafter\def\csname PY@tok@se\endcsname{\let\PY@bf=\textbf\def\PY@tc##1{\textcolor[rgb]{0.73,0.40,0.13}{##1}}}
\expandafter\def\csname PY@tok@sr\endcsname{\def\PY@tc##1{\textcolor[rgb]{0.73,0.40,0.53}{##1}}}
\expandafter\def\csname PY@tok@ss\endcsname{\def\PY@tc##1{\textcolor[rgb]{0.10,0.09,0.49}{##1}}}
\expandafter\def\csname PY@tok@sx\endcsname{\def\PY@tc##1{\textcolor[rgb]{0.00,0.50,0.00}{##1}}}
\expandafter\def\csname PY@tok@m\endcsname{\def\PY@tc##1{\textcolor[rgb]{0.40,0.40,0.40}{##1}}}
\expandafter\def\csname PY@tok@gh\endcsname{\let\PY@bf=\textbf\def\PY@tc##1{\textcolor[rgb]{0.00,0.00,0.50}{##1}}}
\expandafter\def\csname PY@tok@gu\endcsname{\let\PY@bf=\textbf\def\PY@tc##1{\textcolor[rgb]{0.50,0.00,0.50}{##1}}}
\expandafter\def\csname PY@tok@gd\endcsname{\def\PY@tc##1{\textcolor[rgb]{0.63,0.00,0.00}{##1}}}
\expandafter\def\csname PY@tok@gi\endcsname{\def\PY@tc##1{\textcolor[rgb]{0.00,0.63,0.00}{##1}}}
\expandafter\def\csname PY@tok@gr\endcsname{\def\PY@tc##1{\textcolor[rgb]{1.00,0.00,0.00}{##1}}}
\expandafter\def\csname PY@tok@ge\endcsname{\let\PY@it=\textit}
\expandafter\def\csname PY@tok@gs\endcsname{\let\PY@bf=\textbf}
\expandafter\def\csname PY@tok@gp\endcsname{\let\PY@bf=\textbf\def\PY@tc##1{\textcolor[rgb]{0.00,0.00,0.50}{##1}}}
\expandafter\def\csname PY@tok@go\endcsname{\def\PY@tc##1{\textcolor[rgb]{0.53,0.53,0.53}{##1}}}
\expandafter\def\csname PY@tok@gt\endcsname{\def\PY@tc##1{\textcolor[rgb]{0.00,0.27,0.87}{##1}}}
\expandafter\def\csname PY@tok@err\endcsname{\def\PY@bc##1{\setlength{\fboxsep}{0pt}\fcolorbox[rgb]{1.00,0.00,0.00}{1,1,1}{\strut ##1}}}
\expandafter\def\csname PY@tok@kc\endcsname{\let\PY@bf=\textbf\def\PY@tc##1{\textcolor[rgb]{0.00,0.50,0.00}{##1}}}
\expandafter\def\csname PY@tok@kd\endcsname{\let\PY@bf=\textbf\def\PY@tc##1{\textcolor[rgb]{0.00,0.50,0.00}{##1}}}
\expandafter\def\csname PY@tok@kn\endcsname{\let\PY@bf=\textbf\def\PY@tc##1{\textcolor[rgb]{0.00,0.50,0.00}{##1}}}
\expandafter\def\csname PY@tok@kr\endcsname{\let\PY@bf=\textbf\def\PY@tc##1{\textcolor[rgb]{0.00,0.50,0.00}{##1}}}
\expandafter\def\csname PY@tok@bp\endcsname{\def\PY@tc##1{\textcolor[rgb]{0.00,0.50,0.00}{##1}}}
\expandafter\def\csname PY@tok@fm\endcsname{\def\PY@tc##1{\textcolor[rgb]{0.00,0.00,1.00}{##1}}}
\expandafter\def\csname PY@tok@vc\endcsname{\def\PY@tc##1{\textcolor[rgb]{0.10,0.09,0.49}{##1}}}
\expandafter\def\csname PY@tok@vg\endcsname{\def\PY@tc##1{\textcolor[rgb]{0.10,0.09,0.49}{##1}}}
\expandafter\def\csname PY@tok@vi\endcsname{\def\PY@tc##1{\textcolor[rgb]{0.10,0.09,0.49}{##1}}}
\expandafter\def\csname PY@tok@vm\endcsname{\def\PY@tc##1{\textcolor[rgb]{0.10,0.09,0.49}{##1}}}
\expandafter\def\csname PY@tok@sa\endcsname{\def\PY@tc##1{\textcolor[rgb]{0.73,0.13,0.13}{##1}}}
\expandafter\def\csname PY@tok@sb\endcsname{\def\PY@tc##1{\textcolor[rgb]{0.73,0.13,0.13}{##1}}}
\expandafter\def\csname PY@tok@sc\endcsname{\def\PY@tc##1{\textcolor[rgb]{0.73,0.13,0.13}{##1}}}
\expandafter\def\csname PY@tok@dl\endcsname{\def\PY@tc##1{\textcolor[rgb]{0.73,0.13,0.13}{##1}}}
\expandafter\def\csname PY@tok@s2\endcsname{\def\PY@tc##1{\textcolor[rgb]{0.73,0.13,0.13}{##1}}}
\expandafter\def\csname PY@tok@sh\endcsname{\def\PY@tc##1{\textcolor[rgb]{0.73,0.13,0.13}{##1}}}
\expandafter\def\csname PY@tok@s1\endcsname{\def\PY@tc##1{\textcolor[rgb]{0.73,0.13,0.13}{##1}}}
\expandafter\def\csname PY@tok@mb\endcsname{\def\PY@tc##1{\textcolor[rgb]{0.40,0.40,0.40}{##1}}}
\expandafter\def\csname PY@tok@mf\endcsname{\def\PY@tc##1{\textcolor[rgb]{0.40,0.40,0.40}{##1}}}
\expandafter\def\csname PY@tok@mh\endcsname{\def\PY@tc##1{\textcolor[rgb]{0.40,0.40,0.40}{##1}}}
\expandafter\def\csname PY@tok@mi\endcsname{\def\PY@tc##1{\textcolor[rgb]{0.40,0.40,0.40}{##1}}}
\expandafter\def\csname PY@tok@il\endcsname{\def\PY@tc##1{\textcolor[rgb]{0.40,0.40,0.40}{##1}}}
\expandafter\def\csname PY@tok@mo\endcsname{\def\PY@tc##1{\textcolor[rgb]{0.40,0.40,0.40}{##1}}}
\expandafter\def\csname PY@tok@ch\endcsname{\let\PY@it=\textit\def\PY@tc##1{\textcolor[rgb]{0.25,0.50,0.50}{##1}}}
\expandafter\def\csname PY@tok@cm\endcsname{\let\PY@it=\textit\def\PY@tc##1{\textcolor[rgb]{0.25,0.50,0.50}{##1}}}
\expandafter\def\csname PY@tok@cpf\endcsname{\let\PY@it=\textit\def\PY@tc##1{\textcolor[rgb]{0.25,0.50,0.50}{##1}}}
\expandafter\def\csname PY@tok@c1\endcsname{\let\PY@it=\textit\def\PY@tc##1{\textcolor[rgb]{0.25,0.50,0.50}{##1}}}
\expandafter\def\csname PY@tok@cs\endcsname{\let\PY@it=\textit\def\PY@tc##1{\textcolor[rgb]{0.25,0.50,0.50}{##1}}}

\def\PYZbs{\char`\\}
\def\PYZus{\char`\_}
\def\PYZob{\char`\{}
\def\PYZcb{\char`\}}
\def\PYZca{\char`\^}
\def\PYZam{\char`\&}
\def\PYZlt{\char`\<}
\def\PYZgt{\char`\>}
\def\PYZsh{\char`\#}
\def\PYZpc{\char`\%}
\def\PYZdl{\char`\$}
\def\PYZhy{\char`\-}
\def\PYZsq{\char`\'}
\def\PYZdq{\char`\"}
\def\PYZti{\char`\~}
% for compatibility with earlier versions
\def\PYZat{@}
\def\PYZlb{[}
\def\PYZrb{]}
\makeatother


    % Exact colors from NB
    \definecolor{incolor}{rgb}{0.0, 0.0, 0.5}
    \definecolor{outcolor}{rgb}{0.545, 0.0, 0.0}



    
    % Prevent overflowing lines due to hard-to-break entities
    \sloppy 
    % Setup hyperref package
    \hypersetup{
      breaklinks=true,  % so long urls are correctly broken across lines
      colorlinks=true,
      urlcolor=urlcolor,
      linkcolor=linkcolor,
      citecolor=citecolor,
      }
    % Slightly bigger margins than the latex defaults
    
    \geometry{verbose,tmargin=1in,bmargin=1in,lmargin=1in,rmargin=1in}
    
    

    \begin{document}
    
    
    \maketitle
    
    

    
    \paragraph{Erin Witmer}\label{erin-witmer}

CSC 440\\
Homework 3

    \subsection{Chapter 6}\label{chapter-6}

    \paragraph{{[}6.1{]} Suppose you have the set C of all frequent closed
itemsets on a data set D, as well as the support count for each frequen
closed itemset. Describe an algorithm to determine whether a given
itemset X is frequent or not, and the support of X if it is
frequent.}\label{suppose-you-have-the-set-c-of-all-frequent-closed-itemsets-on-a-data-set-d-as-well-as-the-support-count-for-each-frequen-closed-itemset.-describe-an-algorithm-to-determine-whether-a-given-itemset-x-is-frequent-or-not-and-the-support-of-x-if-it-is-frequent.}

    An itemset X is closed in set D if there exists no superset Y such than
Y has the same support count as X in D. Per the text: "The set of closed
frequent itemsets contains complete information regarding the frequent
itemsets." A simple algorithm to determine whether X is frequent or not
and the support is:

\begin{verbatim}
sort set C by support count (high to low)

for each itemset (i) in C:
    
    if X is a subset of C(i):
       return support count for this item 
            
    else:
       next item
    
return X is not frequent if X is not a subset of any itemset in C 
\end{verbatim}

We know that the support count of X if it is a subset of the itemset in
C is the same as the support count in that itemset based on the
definition of the closed itemset. We know it is frequent because C is a
set of all frequent closed itemsets. We know that if X is not a subset
of any itemset in C, it is not frequent.

    \paragraph{{[}6.3{]} The Apriori algorithm makes use of prior knowledge
of subset support
properties.}\label{the-apriori-algorithm-makes-use-of-prior-knowledge-of-subset-support-properties.}

    \paragraph{(a) Prove that all nonempty subsets of a frequent itemset
must also be
frequent.}\label{a-prove-that-all-nonempty-subsets-of-a-frequent-itemset-must-also-be-frequent.}

    An itemset (I) is a frequent itemset if the percentage of transactions
(T) in a database (D) containing that itemset (I) meets or exceeds a
minimum support threshold (min\_sup). That is:

\[ \frac{countOfTransactionsContaining(I)}{totalTransactionsIn(D)} \geq minSupport \\\]if
\[ A \subset I
\] then the count of transactions containing A is greater than or equal
to the count of transactions containing I, because every transaction
containing I, will also contain A. This is true for all non-empty
subsets of I. Support of an itemset never exceeds the support of its
subsets. This is known as the anti-monotone property of support. The
denominator, the count of all transactions, does not change. So if the
numerator (count of transactions containing A) is greater than or equal
to the count of transactions containing I, and the denominator is
constant, then the support level will be greater than or equal to the
support level of I. If I is frequent, or higher than the min support
level, then A will also be frequent.

    \paragraph{(b) Prove that the support of any nonempty subset s′ of
itemset s must be at least as great as the support of
s.}\label{b-prove-that-the-support-of-any-nonempty-subset-s-of-itemset-s-must-be-at-least-as-great-as-the-support-of-s.}

    Again, this can be proven based on the anti-monotone property of
support. Support of an itemset never exceeds the support of its subsets.
The support count of s is the number of times an itemset (s) appears in
the universe of all transactions. For example, if s = \{beer, nuts,
diapers\} then the support count of s is the number of transactions
which contain all three of those items. Any subset of s, for example, s'
= \{beer, nuts\} will occur in all transactions in which s occurs. In
other words, any transaction which contains \{beer, nuts, diapers\} will
also contain all subsets of that itemset \{beer, nuts\}. The support
count of s' will therefore be at least as high as s, and may be higher
if the subset s' occurs in additional transactions.

    \paragraph{(c) Given frequent itemset l and subset s of l, prove that
the confidence of the rule ``s′ -\textgreater{}(l−s′)'' cannot be more
than the confidence of ``s-\textgreater{}(l−s),'' where s′ is a subset
of
s.}\label{c-given-frequent-itemset-l-and-subset-s-of-l-prove-that-the-confidence-of-the-rule-s--ls-cannot-be-more-than-the-confidence-of-s-ls-where-s-is-a-subset-of-s.}

    \[ confidence(s'->l-s')= \frac{supportCount(s' ∪ (l-s'))}{supportCount(s')}\]

\[confidence(s->l-s)= \frac{supportCount(s ∪ (l-s))}{supportCount(s)}\]

    Since we know s' is a subset of s and therefore a subset of l, l-s'
represents the complement to the frequent itemset, l. In union with s',
it equals the frequent itemset. So the count of s' ∪ (l-s') will be the
support count of l. Similarly, l-s represents the complement to the
frequent itemset, and the count of s ∪ (l-s) will be the count of l. So
we know the numerators in the equations above are the same. As proved
above, the support count of s' as a subset of s will always be greater
than or equal to the support count of s. So the denominator in the first
equation above will always be greater than or equal to the denominator
in the second equation, and the value of the first equation will always
be less than or equal to the value of the second equation.

    \paragraph{(d) A partitioning variation of Apriori subdivides the
transactions of a database D into n nonoverlapping partitions. Prove
that any itemset that is frequent in D must be frequent in at least one
partition of
D.}\label{d-a-partitioning-variation-of-apriori-subdivides-the-transactions-of-a-database-d-into-n-nonoverlapping-partitions.-prove-that-any-itemset-that-is-frequent-in-d-must-be-frequent-in-at-least-one-partition-of-d.}

    If a given itemset is frequent in D, then it must be locally frequent in
at least one nonoverlapping partition of D. The itemset is frequent in D
if: \[ \frac{supportCount(I)}{totalTransactionsIn(D)} \geq minSupport \]

If the database is divided up, at least one of the partitions will have
to meet the minimum support for that local partition because if no
partition did, the global support count needed would not be reached. For
example, if there are 100 transactions, and the min\_support is 60\%,
then there must be 60 instances of the itemset in the database. If this
was partitioned into 5 sets of 20 transactions, then the min\_support
count would be 12 for each partition. If C1...C5 \textless{} 12, then
the total support count would be \textless{} 60, and the itemset would
not be frequent. Therefore, at least one of the partitions needs to be
locally frequent if the itemset is frequent.

    \paragraph{{[}6.4{]} Let c be a candidate itemset in Ck generated by the
Apriori algorithm. How many length-(k − 1) subsets do we need to check
in the prune step? Per your previous answer, can you give an improved
version of procedure has infrequent subset in Figure
6.4?}\label{let-c-be-a-candidate-itemset-in-ck-generated-by-the-apriori-algorithm.-how-many-length-k-1-subsets-do-we-need-to-check-in-the-prune-step-per-your-previous-answer-can-you-give-an-improved-version-of-procedure-has-infrequent-subset-in-figure-6.4}

    For every candidate C(k) generated, we need to check k choose k-1
subsets, which is equal to k subsets.

\[ {k \choose k-1} = k\]

For example, if C(k) = \{I1, I2, I3\}, we need to check 3 choose 2, or 3
itemsets: \{I1, I2\}, \{I1, I3\}, \{I2, I3\}. One way to improve the
procedure \texttt{has\_infrequent\_subset()} is to assume that the items
which constructed the candidate set are frequent, so we don't need to
check them. So for example, (k-1) frequent itemsets ABC + ABD = ABCD
(Candidate of k length). Pruning combinations to check are ABC, ABD,
BCD, ACD but it was constructed from ABC, ABD, so those are assumed
frequent, you only have to check BCD and ACD or (k-2) sets.

    \paragraph{{[}6.5{]} Section 6.2.2 describes a method for generating
association rules from frequent itemsets. Propose a more efficient
method. Explain why it is more efficient than the one proposed there.
(Hint: Consider incorporating the properties of Exercises 6.3(b), (c)
into your
design.)}\label{section-6.2.2-describes-a-method-for-generating-association-rules-from-frequent-itemsets.-propose-a-more-efficient-method.-explain-why-it-is-more-efficient-than-the-one-proposed-there.-hint-consider-incorporating-the-properties-of-exercises-6.3b-c-into-your-design.}

    The section proposes the following method for generating association
rules from frequent itemsets: - For each frequent itemset l, generate
all nonempty subsets of l. - For every nonempty subset s of l, output
the rule ``s -\textgreater{} (l − s)'' if support count(l) ≥ min conf,
where min conf is the minimum confidence threshold.

While in general, confidence does not have an anti-monotone property
c(ABC-\textgreater{}D) can be larger or smaller than
c(AB-\textgreater{}D); confidence of rules generated from the same
itemset do have an anti-monotone property. So the example provided in
the text: frequent itemset X = \{I1, I2, I5\}, the resulting association
rules:

\begin{longtable}[]{@{}ll@{}}
\toprule
Rule generated & Confidence\tabularnewline
\midrule
\endhead
\{I1, I2\} -\textgreater{} I5 & confidence = 2/4 = 50\%\tabularnewline
\{I1, I5\} -\textgreater{} I2 & confidence = 2/2 = 100\%\tabularnewline
\{I2, I5\} -\textgreater{} I1 & confidence = 2/2 = 100\%\tabularnewline
I1 -\textgreater{} \{I2, I5\} & confidence = 2/6 = 33\%\tabularnewline
I2 -\textgreater{} \{I1, I5\} & confidence = 2/7 = 29\%\tabularnewline
I5 -\textgreater{} \{I1, I2\} & confidence = 2/2 = 100\%\tabularnewline
\bottomrule
\end{longtable}

The min confidence is 70\%. The first rule generated \{I1, I2\}
-\textgreater{} I5 doesn't meet that threshold. Since confidence is
anti-monotone with respect to the right side of the rule, we know that
any of the rules containing I5 on the right side will not meet the min
confidence threshold either. So we can prune the two rules that contain
I5 on the right: I1 -\textgreater{} \{I2, I5\} and I2 -\textgreater{}
\{I1, I5\} as a result of the anti-monotone property without any
calculations.

On the left hand side, the proof in exercise 6.3(c) shows that for a
frequent itemset I, the confidence of the rule ``s′
-\textgreater{}(l−s′)'' cannot be more than the confidence of
``s-\textgreater{}(l−s),'' where s′ is a subset of s. Once again, the
first rule generated \{I1, I2 -\textgreater{} I5\} doesn't meet that
threshold. Rules with subsets of \{I1, I2\} on the left hand side can
also be pruned. We prune the two rules that contain I1 and/or I2 on the
left: I1 -\textgreater{} \{I2, I5\} and I2 -\textgreater{} \{I1, I5\}
utilizing this enhancement as well.

    \paragraph{{[}6.6{]} A database has five transactions. Let min sup =
60\% and min conf =
80\%.}\label{a-database-has-five-transactions.-let-min-sup-60-and-min-conf-80.}

    \paragraph{(a) Find all frequent itemsets using Apriori and FP-growth,
respectively. Compare the efficiency of the two mining
processes.}\label{a-find-all-frequent-itemsets-using-apriori-and-fp-growth-respectively.-compare-the-efficiency-of-the-two-mining-processes.}

    \begin{longtable}[]{@{}ll@{}}
\toprule
TID & item IDs\tabularnewline
\midrule
\endhead
T100: & {[}'M','O','N','K','E','Y'{]}\tabularnewline
T200: & {[}'D','O','N','K','E','Y'{]}\tabularnewline
T300: & {[}'M','A','K','E'{]}\tabularnewline
T400: & {[}'M','U','C','K','Y'{]}\tabularnewline
T500: & {[}'C','O','O','K','I','E'{]}\tabularnewline
\bottomrule
\end{longtable}

    Apriori (frequent itemsets shown in bold): - Count C1:\\
\{'M': 3, 'O': 3, 'N': 2, 'K': 5, 'E': 4, 'Y': 3, 'D': 1, 'A': 1, 'U':
1, 'C': 2, 'I': 1\}

\begin{itemize}
\item
  Trim infrequent items from C1 to obtain L1:\\
  \textbf{L1: \{'M': 3, 'O': 3, 'K': 5, 'E': 4, 'Y': 3\}}
\item
  Generate C2:\\
  {[}('E', 'K'), ('E', 'M'), ('E', 'O'), ('E', 'Y'), ('K', 'M'), ('K',
  'O'), ('K', 'Y'), ('M', 'O'), ('M', 'Y'), ('O', 'Y'){]}
\item
  Count C2:\\
  \{('E', 'K'): 4, ('E', 'M'): 2, ('E', 'O'): 3, ('E', 'Y'): 2, ('K',
  'M'): 3, ('K', 'O'): 3, ('K', 'Y'): 3, ('M', 'O'): 1, ('M', 'Y'): 2,
  ('O', 'Y'): 2\}
\item
  Trim infrequent items from C2 to obtain L2:\\
  \textbf{L2: \{('E', 'K'): 4, ('E', 'O'): 3, ('K', 'M'): 3, ('K', 'O'):
  3, ('K', 'Y'): 3\}}
\item
  Generate C3:\\
  {[}('E', 'K', 'O'), ('K', 'M', 'O'), ('K', 'M', 'Y'), ('K', 'O',
  'Y'){]}
\item
  Prune C3 based on L2 (infrequent subsets):\\
  {[}('E', 'K', 'O'){]}
\item
  Count C3:\\
  \{('E', 'K', 'O'): 3\}
\item
  Trim infrequent to obtain L3:\\
  \textbf{L3: \{('E', 'K', 'O'): 3\}}
\end{itemize}

    FPGrowth:

\begin{itemize}
\item
  Count C1:\\
  \{'M': 3, 'O': 3, 'N': 2, 'K': 5, 'E': 4, 'Y': 3, 'D': 1, 'A': 1, 'U':
  1, 'C': 2, 'I': 1\}
\item
  Trim infrequent items from C1 to obtain L1 and sort:\\
  \textbf{L1: \{'K': 5, 'E': 4, 'M': 3, 'O': 3, 'Y': 3\}}
\item
  Sort transactions in L order and build FPTree:

\begin{verbatim}
                ROOT
                  |
                K : 5
               /     \
            E : 4   M : 1
           /     \     \
        M : 2   O : 2  Y : 1 
        /          \
    O : 1         Y : 1
     /
  Y : 1 
\end{verbatim}
\item
  Construct conditional pattern base from the tree
\item
  From this build a conditional FPTree for each item, eliminating
  non-frequent paths.
\item
  From this sub tree, generate frequent patterns.
\end{itemize}

\begin{longtable}[]{@{}llll@{}}
\toprule
\begin{minipage}[b]{0.06\columnwidth}\raggedright\strut
Item\strut
\end{minipage} & \begin{minipage}[b]{0.36\columnwidth}\raggedright\strut
Conditional Pattern Base\strut
\end{minipage} & \begin{minipage}[b]{0.22\columnwidth}\raggedright\strut
Conditional FPtree\strut
\end{minipage} & \begin{minipage}[b]{0.25\columnwidth}\raggedright\strut
Frequent Patterns Generated\strut
\end{minipage}\tabularnewline
\midrule
\endhead
\begin{minipage}[t]{0.06\columnwidth}\raggedright\strut
Y\strut
\end{minipage} & \begin{minipage}[t]{0.36\columnwidth}\raggedright\strut
\{K,E,M,O\}: 1, \{K,E,O\}: 1, \{K,M\}: 1\strut
\end{minipage} & \begin{minipage}[t]{0.22\columnwidth}\raggedright\strut
\{K: 3\}\strut
\end{minipage} & \begin{minipage}[t]{0.25\columnwidth}\raggedright\strut
\{KY\}: 3\strut
\end{minipage}\tabularnewline
\begin{minipage}[t]{0.06\columnwidth}\raggedright\strut
O\strut
\end{minipage} & \begin{minipage}[t]{0.36\columnwidth}\raggedright\strut
\{K, E, M\}: 1, \{K, E\}: 2\strut
\end{minipage} & \begin{minipage}[t]{0.22\columnwidth}\raggedright\strut
\{K: 3, E: 3\}\strut
\end{minipage} & \begin{minipage}[t]{0.25\columnwidth}\raggedright\strut
\{KO\}: 3, \{EO\}: 3, \{KEO\}: 3\strut
\end{minipage}\tabularnewline
\begin{minipage}[t]{0.06\columnwidth}\raggedright\strut
M\strut
\end{minipage} & \begin{minipage}[t]{0.36\columnwidth}\raggedright\strut
\{K, E\}: 2, \{K\}: 1\strut
\end{minipage} & \begin{minipage}[t]{0.22\columnwidth}\raggedright\strut
\{K: 3\}\strut
\end{minipage} & \begin{minipage}[t]{0.25\columnwidth}\raggedright\strut
\{KM\}: 3\strut
\end{minipage}\tabularnewline
\begin{minipage}[t]{0.06\columnwidth}\raggedright\strut
E\strut
\end{minipage} & \begin{minipage}[t]{0.36\columnwidth}\raggedright\strut
\{K\}: 4\strut
\end{minipage} & \begin{minipage}[t]{0.22\columnwidth}\raggedright\strut
\{K: 4\}\strut
\end{minipage} & \begin{minipage}[t]{0.25\columnwidth}\raggedright\strut
\{KE\}: 3\strut
\end{minipage}\tabularnewline
\bottomrule
\end{longtable}

    The FPGrowth process is more efficient. Two major costs associated with
the Apriori algorithm are: 1.) the need to generate a huge number of
candidate sets if L1 is long and 2.) the need to scan the database
repeatedly and check a large set of candidates by pattern matching. The
FPGrowth algorithms finds frequent itemsets without candidate
generation. It transforms the problem of finding long frequent patterns
into searching for shorter ones in much smaller conditional databases
recursively and then concatenating the suffix. It uses the least
frequent suffix first, substantially reducing the search cost.

    \paragraph{(b) List all the strong association rules (with support s and
confidence c) matching the following metarule, where X is a variable
representing customers, and item i denotes variables representing items
(e.g., ``A,''
``B,''):}\label{b-list-all-the-strong-association-rules-with-support-s-and-confidence-c-matching-the-following-metarule-where-x-is-a-variable-representing-customers-and-item-i-denotes-variables-representing-items-e.g.-a-b}

\[∀x∈transaction,buys(X,item1)∧buys(X,item2)->buys(X,item3) [s,c]\]

    The following rules are generated. The two transactions shown in bold
match the metarule.

    \begin{longtable}[]{@{}lll@{}}
\toprule
Rule generated & Support & Confidence\tabularnewline
\midrule
\endhead
\{E\} -\textgreater{} \{K\} & support = 4/5 = 80\% & confidence = 4/4 =
100\%\tabularnewline
\{K\} -\textgreater{} \{E\} & support = 4/5 = 100\% & confidence = 4/5 =
80\%\tabularnewline
\{O\} -\textgreater{} \{E\} & support = 3/5 = 60\% & confidence = 3/3 =
100\%\tabularnewline
\{M\} -\textgreater{} \{K\} & support = 3/5 = 60\% & confidence = 3/3 =
100\%\tabularnewline
\{O\} -\textgreater{} \{K\} & support = 3/5 = 60\% & confidence = 3/3 =
100\%\tabularnewline
\{Y\} -\textgreater{} \{K\} & support = 3/5 = 60\% & confidence = 3/3 =
100\%\tabularnewline
\textbf{\{E,O\} -\textgreater{} \{K\}} & \textbf{support = 3/5 = 60\%} &
\textbf{confidence = 3/3 = 100\%}\tabularnewline
\textbf{\{K,O\} -\textgreater{} \{E\}} & \textbf{support = 3/5 = 60\%} &
\textbf{confidence = 3/3 = 100\%}\tabularnewline
\{O\} -\textgreater{} \{K, E\} & support = 3/5 = 60\% & confidence = 3/3
= 100\%\tabularnewline
\bottomrule
\end{longtable}

    \paragraph{{[}6.11{]} Most frequent pattern mining algorithms consider
only distinct items in a transaction. However, multiple occurrences of
an item in the same shopping basket, such as four cakes and three jugs
of milk, can be important in transactional data analysis. How can one
mine frequent itemsets efficiently considering multiple occurrences of
items? Propose modifications to the well-known algorithms, such as
Apriori and FP-growth, to adapt to such a
situation.}\label{most-frequent-pattern-mining-algorithms-consider-only-distinct-items-in-a-transaction.-however-multiple-occurrences-of-an-item-in-the-same-shopping-basket-such-as-four-cakes-and-three-jugs-of-milk-can-be-important-in-transactional-data-analysis.-how-can-one-mine-frequent-itemsets-efficiently-considering-multiple-occurrences-of-items-propose-modifications-to-the-well-known-algorithms-such-as-apriori-and-fp-growth-to-adapt-to-such-a-situation.}

    A simple way to handle this in both the Apriori and FP-growth algorithm
is to treat each instance of a duplicate item as a unique item. So
rather than compressing a transaction with 4 cakes and 3 jugs of milk to
\{cake, milk\} the transaction would be: \{cake\_1, cake\_2, cake\_3,
cake\_4, milk\_1, milk\_2, milk\_3\}. The benefit of this analysis would
be you could analyze association rules like, if you buy 2 cartons of
eggs, how likely are you to buy 2 cartons of milk? The problem with this
is that it dramatically increases the computational complexity of the
analysis because the number of unique items increases. Given d itemsets,
the total number of itemsets = 2\^{}d, and the total possible
association rules = 3\^{}d + 2\^{}(d+1) + 1. So any increase in d will
increase the computational complexity of the analysis exponentially. In
this situation, counting using hash structure would greatly improve the
efficiency of this analysis. It would also be wise to use FPGrowth over
Apriori since L1 would likely increase significantly.


    % Add a bibliography block to the postdoc
    
    
    
    \end{document}
